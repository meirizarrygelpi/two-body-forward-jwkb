\section{Introduction}
%%%%%%%%%%%%%%%%%%%%%%%%%%%%%%%%%%%%%%%%%%%%%%%%%%%%%%%%%%%%%%%%%%%%%%%%%%%%%%%%%%%%%%%%%
Scattering amplitudes are important quantities that bridge theoretical and experimental results. Many tools for the computation of amplitudes have been developed over the past decades. Exact perturbative amplitudes (i.e. tree-level, one-loop, two-loops, etc.) with generic kinematic data can be computed for many theories (see \cite{Elvang:2013cua} for a review). There are also tools for the computation of non-perturbative amplitudes, but these typically involve adding an infinite set of perturbative amplitudes with restricted kinematic data (e.g. Regge limit, fixed-angle limit, etc.).

Some years ago, a program was started by Alday \& Maldacena \cite{Alday:2007hr} to study fixed-angle scattering of gluons in planar $\mathcal{N} = 4$ super Yang-Mills theory at strong coupling. The computation of the scattering amplitude of gluons at strong coupling translates, via the AdS/CFT correspondence, to the computation of a semiclassical amplitude for bosonic strings propagating in $AdS_{5}$. The resulting four-point amplitude at strong coupling agrees with an ansatz of Bern, Dixon \& Smirnov \cite{Bern:2005iz} for the four-point planar MHV scattering amplitude. Other results from this program include dual conformal symmetry, the Yangian and the relation to Wilson loops \cite{Alday:2008yw,Drummond:2010km,Alday:2010kn}.

It is easy to wonder if analogous results can be obtained for less restrictive theories (i.e. other than planar, superconformal gauge theories with string duals). Halpern \& Siegel \cite{HalpernSiegel} found that the (semiclassical) JWKB approximation of certain quantum mechanical systems (e.g. systems of particles) leads to a strong-coupling expansion. In nonrelativistic quantum mechanics, the semiclassical limit of the Feynman path integral gives
\begin{equation}
	\hbar \rightarrow 0: \qquad \int\limits_{\mathbf{x}_{I}}^{\mathbf{x}_{O}} \mathrm{D}\mathbf{q}(t) \, \exp{\left( - \frac{i}{\hbar} S[\mathbf{q}] \right)} \approx \sqrt{\det{(\mathbf{V})}} \exp{\left( - \frac{i}{\hbar} \Sigma \right)} \label{FeynmanPath}
\end{equation}
where
\begin{equation}
	\Sigma \equiv S[\mathbf{q}_{*}], \qquad \mathbf{V} \equiv - \frac{i}{\hbar} \frac{\partial^{2} \Sigma}{\partial \mathbf{x}_{I} \partial \mathbf{x}_{O}}
\end{equation}
Here $\mathbf{q}_{*}(t)$ is a solution of the classical equations of motion with boundary conditions $\mathbf{q}_{*}(t_{I}) = \mathbf{x}_{I}$ and $\mathbf{q}_{*}(t_{O}) = \mathbf{x}_{O}$. The right-hand side of (\ref{FeynmanPath}) is sometimes known as the Van Vleck-Morette approximation to the path integral \cite{VanVleck,CartierMorette}. We use the relativistic sister of this approximation to study four-point near-forward scattering (i.e. small-angle) in a system with two non-identical, heavy, scalar particles. When the particles are coupled via the exchange of massless spin $N$ quanta, in $D$ spacetime dimensions the scattering amplitude takes the form
\begin{equation}
	\mathcal{A} = \delta(P) \left[ \frac{\mathcal{K}_{N}(s)}{\rho_{N}(s)} \right] \int \mathrm{d}B_{12} \, \exp{\left[ - i B_{12} \cdot P_{12} + \beta_{N} \rho_{N}(s) \Gamma(\Delta - 1) \left( \frac{2}{B_{12}^{2}} \right)^{(\Delta - 1)} \right]} \label{AmpMassless}
\end{equation}
where $P_{12}^{2} = -t$, $\Delta = (D - 2)/2$, $\beta_{N}$ is a coupling, and the integral over $B_{12}$ is over a volume in $D - 2$ dimensions. This amplitude has the familiar ``eikonal'' form, which was derived long ago by adding perturbative contributions from all ladder diagrams \cite{ChengWuPRL,AbarbItzyk,LevySucher1,ChangMa}.

In \S\ref{sec3} we derive (\ref{AmpMassless}) without any reference to (perturbative) Feynman diagrams in an attempt to make its non-perturbative nature explicit from the beginning. In \S\ref{sec4} and \S\ref{sec5} we evaluate this amplitude in $D = 3$ (i.e. $\Delta = 1/2$) and $D = 4$ (i.e. $\Delta = 1$), respectively. In both cases we consider interactions mediated by massless spin $0$, $1$ and $2$ quanta, and we find amplitudes that exhibit bound state singularities. In four spacetime dimensions we find the familiar spectrum with an infinite number of bound states \cite{BIZJ,KabatOrtiz,Dittrich}, but in three spacetime dimensions we find an amplitude with only one bound-state-like singularity, even for the exchange of non-propagating three-dimensional massless spin $2$ quanta.

Then, in \S\ref{sec6} we consider the exchange of massive, spin $0$ quanta in $D = 3$ and $D = 5$ (such that $D - 2 = 1$ and $3$, which are the cases when the long-distance propagator is exact). For the massive exchange in $D = 3$ we find an amplitude that exhibits an infinite number of singularities, but these correspond to the multi-mass branch points (and does not involve the branch cut continuum). Alas, in $D = 5$ it becomes necessary to perform a perturbative expansion in the coupling. We find the expected tree-level amplitude, along with a finite one-loop amplitude and divergent higher-loop amplitudes.

In the next section we begin by introducing the approximation that we use, the forward-JWKB approximation, and contrasting it with another commonly-used approximation, the Regge limit.