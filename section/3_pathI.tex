\section{Path Integrals\label{sec3}}
%%%%%%%%%%%%%%%%%%%%%%%%%%%%%%%%%%%%%%%%%%%%%%%%%%%%%%%%%%%%%%%%%%%%%%%%%%%%%%%
We treat the matter quanta as particles (i.e. not fields). In the models that we study, each particle couples to a mediating field $H_{N}$ with spin $N \geq 0$. We will consider massless mediating fields with $N = 0$, $1$ and $2$, and a massive mediating field with $N = 0$. The action functional for the $\Phi_{1}$ and $\Phi_{2}$ particles in (\ref{ScaProc}) has the form
\begin{equation}
	S_{\text{part}}[ q_{1}, q_{2}, H_{N} ] = S_{\text{free}}[ q_{1}, q_{2} ] + S_{\text{int}}[ q_{1}, q_{2}, H_{N} ].
	\label{SP}
\end{equation}
Here, $S_{\text{free}}$ contains the (free) worldline-gauge-fixed kinetic terms,
\begin{equation}
	S_{\text{free}}[ q_{1}, q_{2} ] = \frac{1}{2}\int \mathrm{d}\tau_{1} \left[-\dot{q}_{1}^{2} + M_{1}^{2} \right] + \frac{1}{2} \int \mathrm{d}\tau_{2} \left[- \dot{q}_{2}^{2} + M_{2}^{2} \right];
\end{equation}
and $S_{\text{int}}$ contains the worldline-gauge-fixed terms with the coupling to the field $H_{N}$,
\begin{equation}
\begin{split}
	S_{\text{int}}[ q_{1}, q_{2}, H_{N} ] = {}& \frac{g_{N}}{N!} \int \mathrm{d} \tau_{1} \, \dot{q}_{1}^{a_{1}} \cdots \dot{q}_{1}^{a_{N}} (H_{N}[q_{1}(\tau_{1})])_{a_{1} \cdots a_{N}} \\
	&+ \frac{g_{N}}{N!} \int \mathrm{d} \tau_{2} \, \dot{q}_{2}^{b_{1}} \cdots \dot{q}_{2}^{b_{N}} (H_{N}[q_{2}(\tau_{2})])_{b_{1} \cdots b_{N}};
\end{split}
\label{Sint}
\end{equation}
where $g_{N}$ is a dimensionful coupling, and the $M_{i}$ are ``internal'' worldline masses which are a priori different from the external masses $m_{i}$. The field $H_{N}$ is totally symmetric in the $N$ spacetime indices.

The mediating field $H_{N}$ is made dynamical by adding a kinetic term to the particle action (\ref{SP}),
\begin{equation}
	S[ q_{1}, q_{2}, H_{N} ] = S_{\text{kin}}[H_{N}] + S_{\text{part}}[ q_{1}, q_{2}, H_{N} ].
\end{equation}
We will mostly consider the free, massless case. That is, when $N = 0$ we have a free, massless, scalar field $H_{0}$; when $N = 1$ we have an abelian, vector field $(H_{1})_{a}$ (photon); and when $N = 2$ we have a linearized, symmetric, tensor field $(H_{2})_{ab}$ (massless Fierz-Pauli graviton). After an appropriate gauge-fixing (we use the analog of the Fermi-Feynman gauge), the kinetic term in $S_{\text{kin}}$ takes the form
\begin{equation}
	S_{\text{kin}}[H_{N}] = \frac{1}{2} \int \int \mathrm{d}x \mathrm{d}y \left( [H_{N}(x)]_{a_{1} \cdots a_{N}} [K_{N}(x|y)]^{a_{1} \cdots a_{N} b_{1} \cdots b_{N}} [H_{N}(y)]_{b_{1} \cdots b_{N}} \right);
\end{equation}
where the free, massless, spin $N$ gauge-fixed kinetic operator $K_{N}$ is given by
\begin{equation}
	[K_{N}(x|y)]^{a_{1} \cdots a_{N} b_{1} \cdots b_{N}} = (\kappa_{N})^{a_{1} \cdots a_{N} b_{1} \cdots b_{N}} K_{0}(x|y);
\end{equation}
with $\kappa_{N}$ a constant tensor that is separately totally symmetric in the $a_{j}$ and $b_{k}$ indices, and $K_{0}$ is the free, massless, scalar kinetic operator,
\begin{equation}
	K_{0}(x|y) = \delta(x - y) \left(- \frac{1}{2} \partial^{2} \right).
\end{equation}
%%%%%%%%%%%%%%%%%%%%%%%%%%%%%%%%%%%%%%%%%%%%%%%%%%%%%%%%%%%%%%%%%%%%%%%%%%%%%%%
\subsection{Forward-JWKB Path Integrals}
%%%%%%%%%%%%%%%%%%%%%%%%%%%%%%%%%%%%%%%%%%%%%%%%%%%%%%%%%%%%%%%%%%%%%%%%%%%%%%%
The system $H_{N} + \Phi_{1} + \Phi_{2}$ is described by the un-integrated quantum path integral (i.e. a path integral that is dependent on the modulus of each worldline):
\begin{equation}
	\mathcal{F}_{T}(3, 4|1, 2) = \int \widehat{\mathrm{D}}H_{N}(x) \int\limits_{x_{1}}^{x_{3}} \mathrm{D}q_{1}(\tau_{1}) \int\limits_{x_{2}}^{x_{4}} \mathrm{D}q_{2}(\tau_{2}) \exp{\left( - i S\left[ q_{1}, q_{2}, H_{N} \right] \right)}.
\end{equation}
The functional measure over $H_{N}$ is normalized such that
\begin{equation}
	\int \widehat{\mathrm{D}}H_{N}(x) \exp{\left(- i S_{\text{kin}}[H_{N}] \right)} = 1.
\end{equation}
That is, if $g_{N} = 0$, then $\mathcal{F}_{T}$ reduces to the product of two un-integrated free, massive, scalar Green functions.

After writing $S_{\text{int}}$ in (\ref{Sint}) as a spacetime volume integral,
\begin{equation}
	S_{\text{int}}[ q_{1}, q_{2}, H_{N} ] = \int \mathrm{d}x \left[ J_{1}(x) \cdot H_{N}(x) + J_{2}(x) \cdot H_{N}(x) \right];
\end{equation}
with the aid of sources $J_{1}$ and $J_{2}$ given by
\begin{align}
	[J_{1}(x)]^{a_{1} \cdots a_{N}} &\equiv \frac{g_{N}}{N!} \int \mathrm{d} \tau_{1} \, \dot{q}_{1}^{a_{1}} \cdots \dot{q}_{1}^{a_{N}} \delta[x - q_{1}(\tau_{1})], \\
	[J_{2}(x)]^{b_{1} \cdots b_{N}} &\equiv \frac{g_{N}}{N!} \int \mathrm{d} \tau_{2} \, \dot{q}_{2}^{b_{1}} \cdots \dot{q}_{2}^{b_{N}} \delta[x - q_{2}(\tau_{2})];
\end{align}
we perform the functional integral over $H_{N}$ and find
\begin{equation}
	\mathcal{F}_{T}(3, 4|1, 2) = \int\limits_{x_{1}}^{x_{3}} \mathrm{D}q_{1}(\tau_{1}) \int\limits_{x_{2}}^{x_{4}} \mathrm{D}q_{2}(\tau_{2}) \exp{\left( - i S_{\text{eff}}\left[ q_{1}, q_{2}\right] \right)};
	\label{FSeff}
\end{equation}
with the functional $S_{\text{eff}}$ given by
\begin{equation}
	S_{\text{eff}}\left[ q_{1}, q_{2}\right] \equiv S_{\text{free}}\left[ q_{1}, q_{2}\right] - \frac{1}{2} \int \int \mathrm{d}x \mathrm{d}y \left[ J_{1}(x) + J_{2}(x) \right] \cdot G_{N}(x|y) \cdot \left[ J_{1}(y) + J_{2}(y) \right].
	\label{Seff}
\end{equation}
Here $G_{N} = (K_{N})^{-1}$ is the free, massless, spin $N$ gauge-fixed Green function,
\begin{equation}
	[G_{N}(x|y)]_{a_{1} \cdots a_{N} b_{1} \cdots b_{N}} = -i (\nu_{N})_{a_{1} \cdots a_{N} b_{1} \cdots b_{N}} \Gamma\left( \Delta \right) \left[ \frac{2}{(x - y)^{2}} \right]^{\Delta}, \qquad \Delta \equiv \frac{D - 2}{2};
	\label{GN}
\end{equation}
with $\nu_{N}$ a constant tensor that satisfies
\begin{equation}
	(\nu_{N})_{a_{1} \cdots a_{N} c_{1} \cdots c_{N}} (\kappa_{N})^{c_{1} \cdots c_{N} b_{1} \cdots b_{N}} = \frac{1}{N!} \left( \delta_{a_{1}}{}^{b_{1}} \cdots \delta_{a_{N}}{}^{b_{N}} + \text{ permutations} \right).
\end{equation}

The outcome of integrating over $H_{N}$ is the appearance of ``potential'' terms that describe the effective interactions between the $\Phi_{1}$ and $\Phi_{2}$ particles. Indeed, we write $S_{\text{eff}}$ in (\ref{Seff}) as
\begin{equation}
	S_{\text{eff}}[q_{1}, q_{2}] = S_{\text{free}}[q_{1}, q_{2}] - S_{\text{one}}[q_{1}, q_{2}] - S_{\text{two}}[q_{1}, q_{2}];
\end{equation}
with $S_{\text{one}}$ containing self-interaction terms,
\begin{equation}
\begin{split}
	S_{\text{one}} = {}& \frac{g_{N}^{2}}{2 (N!)^{2}} \int \int \mathrm{d}\tau_{1} \mathrm{d}\sigma_{1} \left[ \dot{q}_{1}(\tau_{1}) \cdots \dot{q}_{1} \cdot G_{N}[q_{1}(\tau_{1}) | q_{1}(\sigma_{1})] \cdot \dot{q}_{1}(\sigma_{1}) \cdots \dot{q}_{1} \right] \\
	&+ \frac{g_{N}^{2}}{2 (N!)^{2}} \int \int \mathrm{d}\tau_{2} \mathrm{d}\sigma_{2} \left[ \dot{q}_{2}(\tau_{2}) \cdots \dot{q}_{2} \cdot G_{N}[q_{2}(\tau_{2}) | q_{2}(\sigma_{2})] \cdot \dot{q}_{2}(\sigma_{2}) \cdots \dot{q}_{2} \right];
\end{split}
\label{S10}
\end{equation}
and $S_{\text{two}}$ containing a two-body interaction term,
\begin{equation}
	S_{\text{two}} = \frac{g_{N}^{2}}{(N!)^{2}} \int \int \mathrm{d}\tau_{1} \mathrm{d}\tau_{2} \left[ \dot{q}_{1}(\tau_{1}) \cdots \dot{q}_{1} \cdot G_{N}[q_{1}(\tau_{1}) | q_{2}(\tau_{2})] \cdot \dot{q}_{2}(\tau_{2}) \cdots \dot{q}_{2} \right].
	\label{S20}
\end{equation}
One can think of the first term in $S_{\text{one}}$ as summing over contributions that involve linking a point $q_{1}(\tau_{1})$ to a point $q_{1}(\sigma_{1})$ with a propagator $G_{N}$. Both of these points are on the $\Phi_{1}$ worldline. Similarly, the second term in $S_{\text{one}}$ involves linking two points on the $\Phi_{2}$ worldline. On the other hand, $S_{\text{two}}$ can be understood as summing over contributions that involve linking a point $q_{1}(\tau_{1})$ on the $\Phi_{1}$ worldline to a point $q_{2}(\tau_{2})$ on the $\Phi_{2}$ worldline.

At this stage, the discussion is general and exact. In what follows \textbf{we ignore the self-interaction terms}. When this is done, $\mathcal{F}_{T}$ in (\ref{FSeff}) takes the form
\begin{equation}
	\mathcal{F}_{T}(3, 4|1, 2) = \int\limits_{x_{1}}^{x_{3}} \mathrm{D}q_{1}(\tau_{1}) \int\limits_{x_{2}}^{x_{4}} \mathrm{D}q_{2}(\tau_{2}) \exp{\left( - i S_{\text{free}} \right)} \exp{\left( i S_{\text{two}} \right)}.
	\label{320}
\end{equation}
We \textit{could} expand the second exponential in (\ref{320}) as a perturbative expansion in $g_{N}$,
\begin{equation}
	\exp{\left( i S_{\text{two}} \right)} = 1 + i S_{\text{two}} + \frac{1}{2!} (i S_{\text{two}})^{2} + \frac{1}{3!} (i S_{\text{two}})^{3} + \cdots;
	\label{321}
\end{equation}
and evaluate each contribution. The first term in (\ref{321}) is of order $g_{N}^{0}$ and not very interesting. The second term is proportional to $S_{\text{two}}$ and thus of order $g_{N}^{2}$. It involves a sum over all possible ways to connect a point on the $\Phi_{1}$ worldline to a point on the $\Phi_{2}$ worldline with a $G_{N}$ propagator. This is a tree-level contribution. The third term involves
\begin{equation}
	(i S_{\text{two}})^{2} \sim g_{N}^{4} \int \int \mathrm{d}\tau_{1} \mathrm{d}\tau_{2} \int \int \mathrm{d}\sigma_{1} \mathrm{d}\sigma_{2} (\cdots) G_{N}[q_{1}(\tau_{1}) | q_{2}(\tau_{2})] G_{N}[q_{1}(\sigma_{1}) | q_{2}(\sigma_{2})].
\end{equation}
This can be identified with a one-loop contribution, but since the double integration scans all possible orderings of the worldline coordinates, it accounts for both box-like and crossed box-like contributions. Similarly, the fourth term in (\ref{321}) corresponds to the two-loops contribution, and contains the double box, the crossed double box and other non-planar contributions. Thus, we have learned that $\mathcal{F}_{T}$ in (\ref{320}) contains all perturbative contributions arising from generalized ladder diagrams. Also, it follows that the contributions to the scattering amplitude from $\mathcal{F}_{T}$ are un-truncated and we need to perform some truncation.

We now incorporate the forward-JWKB approximation into our analysis. Since the forward-JWKB approximation is a combination of the forward approximation and the semiclassical approximation, in the forward-JWKB approximation the un-integrated quantum path integral $\mathcal{F}_{T}$ takes the form
\begin{equation}
	\mathcal{F}_{T}(3, 4|1, 2) \longrightarrow \mathcal{G}_{T}(3, 4|1, 2) = \sqrt{-\det{(V)}} \exp{\left( -i \Sigma \right)}.
\end{equation}
where the function $\Sigma$ is the value of $S_{\text{eff}}$ evaluated at the forward paths $f_{1}$ and $f_{2}$,
\begin{equation}
	\Sigma \equiv S_{\text{eff}} [f_{1}, f_{2}];
\end{equation}
and the matrix $V$ is given by
\begin{equation}
	V \equiv \begin{pmatrix}
	V_{1 3} & V_{2 3} \\
	V_{1 4} & V_{2 4}
	\end{pmatrix}, \qquad V_{jk} \equiv - i \frac{\partial \Sigma}{\partial x_{j} \partial x_{k}}.
\end{equation}
The forward paths describe particles that are moving along straight paths in spacetime with fixed spacetime speed:
\begin{equation}
\begin{split}
	f_{1}(\tau_{1}) &= \frac{x_{1} + x_{3}}{2} + \left( \frac{\tau_{1}}{T_{1}} \right) \left(x_{3} - x_{1} \right), \qquad { - \frac{T_{1}}{2} } < \tau_{1} < \frac{T_{1}}{2}; \\
	f_{2}(\tau_{2}) &= \frac{x_{2} + x_{4}}{2} + \left( \frac{\tau_{2}}{T_{2}} \right) \left(x_{4} - x_{2} \right), \qquad { - \frac{T_{2}}{2} } < \tau_{2} < \frac{T_{2}}{2}.
\end{split}
\label{fpaths}
\end{equation}
Here $T_{1}$ and $T_{2}$ are the moduli of the $\Phi_{1}$ and $\Phi_{2}$ worldlines, respectively. The form of $\mathcal{G}_{T}$ can be recognized as the (relativistic) two-body generalization of the Van Vleck-Morette kernel \cite{VanVleck,CartierMorette} specialized to the forward paths (i.e. we do not solve for the true classical paths). For this reason we refer to $\mathcal{G}_{T}$ as the un-integrated forward-JWKB kernel. In the rest of this section we evaluate $\mathcal{G}_{T}$ and relate it to the scattering amplitude.
%%%%%%%%%%%%%%%%%%%%%%%%%%%%%%%%%%%%%%%%%%%%%%%%%%%%%%%%%%%%%%%%%%%%%%%%%%%%%%%
\subsubsection{Forward Van Vleck Function}
%%%%%%%%%%%%%%%%%%%%%%%%%%%%%%%%%%%%%%%%%%%%%%%%%%%%%%%%%%%%%%%%%%%%%%%%%%%%%%%
At the forward paths (\ref{fpaths}), the free part of $S_{\text{eff}}$ gives
\begin{equation}
	\Sigma_{\text{free}} \equiv S_{\text{free}}[f_{1}, f_{2}] = - \frac{1}{2 T_{1}} x_{3 1}^{2} + \frac{M_{1}^{2} T_{1}}{2} - \frac{1}{2 T_{2}} x_{4 2}^{2} + \frac{M_{2}^{2} T_{2}}{2}, \qquad x_{jk} \equiv x_{j} - x_{k}.
	\label{SigFree}
\end{equation}
Recall that we are dropping the self-interactions and keeping the two-body interaction. Evaluating $S_{\text{two}}$ at the forward paths gives $\Sigma_{\text{two}} \equiv S_{\text{two}}[f_{1}, f_{2}]$, i.e. 
\begin{equation}
	\Sigma_{\text{two}} = \frac{g_{N}^{2}}{(N!)^{2}} \int \int \mathrm{d}\tau_{1} \mathrm{d}\tau_{2} \left[ \dot{f}_{1}(\tau_{1}) \cdots \dot{f}_{1} \cdot G_{N}[f_{1}(\tau_{1}) | f_{2}(\tau_{2})] \cdot \dot{f}_{2}(\tau_{2}) \cdots \dot{f}_{2} \right].
\end{equation}
Note that the forward paths have constant slope:
\begin{equation}
	\dot{f}_{1} = \frac{x_{31}}{T_{1}} \equiv k_{31}, \qquad \dot{f}_{2} = \frac{x_{42}}{T_{2}} \equiv k_{42}.
	\label{slope}
\end{equation}
Using (\ref{GN}) and (\ref{slope}), we write
\begin{equation}
	\dot{f}_{1}(\tau_{1}) \cdots \dot{f}_{1} \cdot G_{N}[f_{1}(\tau_{1}) | f_{2}(\tau_{2})] \cdot \dot{f}_{2}(\tau_{2}) \cdots \dot{f}_{2} = \mathcal{K}_{N} G_{0}[f_{1}(\tau_{1}) | f_{2}(\tau_{2})];
\end{equation}
where $G_{0}$ is the massless, scalar Green function and the constant, scalar factor $\mathcal{K}_{N}$ consists of contractions of $N$ copies of $k_{3 1}$ and $k_{4 2}$ with the constant tensor $\nu_{N}$:
\begin{equation}
	\mathcal{K}_{N} = \left( \frac{1}{T_{1} T_{2}} \right)^{N} \left(x_{3 1} \cdots x_{3 1} \right) \cdot \nu_{N} \cdot \left(x_{4 2} \cdots x_{4 2}\right) = \left(k_{3 1} \cdots k_{3 1} \right) \cdot \nu_{N} \cdot \left(k_{4 2} \cdots k_{4 2}\right).
\end{equation}
With the forward paths (\ref{fpaths}), we have
\begin{equation}
	f_{1}(\tau_{1}) - f_{2}(\tau_{2}) = X_{1 2} + \left( \frac{\tau_{1}}{T_{1}} \right) x_{3 1} - \left( \frac{\tau_{2}}{T_{2}} \right) x_{4 2};
\end{equation}
where we have introduced
\begin{equation}
	X_{1 2} \equiv \frac{x_{1} - x_{2} + x_{3} - x_{4}}{2}.
\end{equation}
Note that $X_{1 2}$ is the vector average of the separation of the incoming particles (given by the vector $x_{1} - x_{2}$) and the separation of the outgoing particles (given by the vector $x_{3} - x_{4}$). That is, $X_{12}$ is incoming/outgoing symmetric. We can now write $\Sigma_{\text{two}}$ as
\begin{equation}
	\Sigma_{\text{two}} = -i\frac{g_{N}^{2}}{(N!)^{2}} T_{1} T_{2} \mathcal{K}_{N} \Upsilon_{\text{two}};
\end{equation}
with
\begin{equation}
	\Upsilon_{\text{two}} \equiv \Gamma(\Delta) \int\limits_{-1/2}^{1/2}\mathrm{d}u_{1} \int\limits_{-1/2}^{1/2}\mathrm{d}u_{2} \left[ \frac{2}{(X_{1 2} + u_{1} x_{3 1} - u_{2} x_{4 2})^{2}} \right]^{\Delta}, \qquad \Delta \equiv \frac{D - 2}{2}.
\end{equation}
(We have changed variables from $(\tau_{1}, \tau_{2})$ to $(u_{1}, u_{2})$, which are dimensionless). It is convenient to introduce a Schwinger parameter $T$ and write
\begin{equation}
	\Upsilon_{\text{two}} = \int\limits_{-1/2}^{1/2}\mathrm{d}u_{1} \int\limits_{-1/2}^{1/2}\mathrm{d}u_{2} \int\limits_{0}^{\infty}\mathrm{d}T \left( \frac{1}{T} \right)^{(\Delta + 1)} \exp{\left[- \frac{1}{2 T} (X_{1 2} + u_{1} x_{3 1} - u_{2} x_{4 2})^{2} \right]}.
	\label{UpSchw}
\end{equation}
The path difference $f_{1}(u_{1}) - f_{2}(u_{2}) = X_{1 2} + u_{1} x_{3 1} - u_{2} x_{4 2}$ describes the separation between the particles during the scattering process. The conjugate momentum to this separation is the momentum transfer. In the forward-JWKB approximation, the momentum transfer is very small compared to the masses or the center-of-momentum energy. By Fourier-Heisenberg conjugacy, this means that the separation between the particles is always very large compared to the displacement of each particle. Thus, in the forward-JWKB approximation the $(u_{1}, u_{2})$ integral in (\ref{UpSchw}) is dominated by the contribution from the critical point of the expression in the exponent:
\begin{align}
	\bar{u}_{1} &= - \left[ \frac{x_{42}^{2} (X_{12} \cdot x_{31}) - (X_{12} \cdot x_{42})(x_{31} \cdot x_{42})}{x_{31}^{2} x_{42}^{2} - (x_{31} \cdot x_{42})^{2}} \right], \\
	\bar{u}_{2} &= + \left[ \frac{x_{31}^{2} (X_{12} \cdot x_{42}) - (X_{12} \cdot x_{31})(x_{31} \cdot x_{42})}{x_{31}^{2} x_{42}^{2} - (x_{31} \cdot x_{42})^{2}} \right].
\end{align}
At this critical point, we find
\begin{equation}
	B_{12} \equiv f_{1}(\bar{u}_{1}) - f_{2}(\bar{u}_{2}) = X_{1 2} + \bar{u}_{1} x_{3 1} - \bar{u}_{2} x_{4 2};
\end{equation}
which satisfies $B_{12} \cdot x_{31} = 0$ and $B_{12} \cdot x_{42} = 0$. That is, $B_{12}$ is the projection of $X_{12}$ to the subspace that is orthogonal to $x_{31}$ and $x_{42}$. In the forward-JWKB approximation we find
\begin{align}
	\Upsilon_{\text{two}} &\approx \frac{2 \pi}{\sqrt{x_{31}^{2} x_{42}^{2} - (x_{31} \cdot x_{42})^{2}}} \int\limits_{0}^{\infty}\mathrm{d}T \left( \frac{1}{T} \right)^{\Delta} \exp{\left[- \frac{1}{2 T} B_{1 2}^{2} \right]} \nonumber \\
	&= \frac{2 \pi}{\sqrt{x_{31}^{2} x_{42}^{2} - (x_{31} \cdot x_{42})^{2}}} \Gamma(\Delta - 1) \left( \frac{2}{B_{1 2}^{2}} \right)^{(\Delta - 1)};
\end{align}
and hence $\Sigma_{\text{two}}$ gives
\begin{equation}
	\Sigma_{\text{two}} \approx -i \left( \frac{2 \pi g_{N}^{2}}{(N!)^{2}} \right) \left[ \frac{T_{1} T_{2} \mathcal{K}_{N}}{\sqrt{x_{31}^{2} x_{42}^{2} - (x_{31} \cdot x_{42})^{2}}} \right] \Gamma(\Delta - 1) \left( \frac{2}{B_{1 2}^{2}} \right)^{(\Delta - 1)}.
\end{equation}
Note that this expression is divergent when $\Delta = 1$ (i.e. when $D = 4$). In order to keep things compact, we introduce the coupling
\begin{equation}
	\beta_{N} \equiv \frac{2 \pi g_{N}^{2}}{(N!)^{2}};
	\label{betaN}
\end{equation}
and the function
\begin{equation}
	\rho_{N} \equiv \frac{T_{1} T_{2} \mathcal{K}_{N}}{\sqrt{x_{31}^{2} x_{42}^{2} - (x_{31} \cdot x_{42})^{2}}} = \frac{\mathcal{K}_{N}}{\sqrt{k_{31}^{2} k_{42}^{2} - (k_{31} \cdot k_{42})^{2}}}.
\end{equation}
When $\rho_{N}$ is written in terms of $k_{31}$ and $k_{42}$, there are no explicit factors of $T_{1}$ and $T_{2}$.
%%%%%%%%%%%%%%%%%%%%%%%%%%%%%%%%%%%%%%%%%%%%%%%%%%%%%%%%%%%%%%%%%%%%%%%%%%%%%%%
\subsubsection{Forward Van Vleck Matrix}
%%%%%%%%%%%%%%%%%%%%%%%%%%%%%%%%%%%%%%%%%%%%%%%%%%%%%%%%%%%%%%%%%%%%%%%%%%%%%%%
Since the Van Vleck function $\Sigma$ has the form $\Sigma_{\text{free}} - \Sigma_{\text{two}}$, we write the Van Vleck matrix $V$ also in the form $V_{\text{free}} - V_{\text{two}}$ with
\begin{equation}
	V_{\text{free}} = \begin{pmatrix}
	u_{1 3} & u_{2 3} \\
	u_{1 4} & u_{2 4}
	\end{pmatrix}, \quad u_{jk} \equiv - i \frac{\partial \Sigma_{\text{free}}}{\partial x_{j} \partial x_{k}}; \quad V_{\text{two}} = \begin{pmatrix}
	v_{1 3} & v_{2 3} \\
	v_{1 4} & v_{2 4}
	\end{pmatrix}, \quad v_{jk} \equiv i \frac{\partial \Sigma_{\text{two}}}{\partial x_{j} \partial x_{k}}.
\end{equation}
The determinant of $V$ can be written as
\begin{equation}
	\det{(V)} = \det(V_{\text{free}} - V_{\text{two}}) = \det{(I - W)} \det{(V_{\text{free}})}, \qquad W \equiv V_{\text{two}} \cdot (V_{\text{free}})^{-1}.
\end{equation}
Hence, the square root of the determinant is
\begin{equation}
	\sqrt{-\det{(V)}} = \sqrt{-\det{(V_{\text{free}})}} \exp{\left[ - \sum_{n = 1}^{\infty} \frac{1}{2 n} \operatorname{tr}{(W^{n})} \right]}.
	\label{sqrtV}
\end{equation}
Using $\Sigma_{\text{free}}$ from (\ref{SigFree}), it is easy to show that
\begin{equation}
\begin{split}
	(u_{13})_{ab} = \left( - \frac{i}{T_{1}} \right) \eta_{ab}, &\qquad (u_{23})_{ab} = 0; \\
	(u_{14})_{ab} = 0, &\qquad (u_{24})_{ab} = \left( - \frac{i}{T_{2}} \right) \eta_{ab};
\end{split}
\end{equation}
and thus
\begin{equation}
	\sqrt{-\det{(V_{\text{free}})}} = \left( - \frac{i}{T_{1}} \right)^{D/2} \left( - \frac{i}{T_{2}} \right)^{D/2}.
\end{equation}
We can think of the terms with traces of $W$ in (\ref{sqrtV}) as corrections to $\Sigma$, since they appear inside an exponential too. If we compute them, we find that they involve powers of $B_{12}^{2}$ that are more negative than the power in $\Sigma_{\text{two}}$. In the forward-JWKB approximation $B_{1 2}^{2}$ is large, so we keep the dominant contribution from $\Sigma_{\text{two}}$ and drop all terms with traces of $W$. Hence,
\begin{equation}
	\sqrt{-\det{(V)}} \approx \sqrt{-\det{(V_{\text{free}})}}.
\end{equation}
This step might seem drastic, but as we shall see, the end result will justify our means.
%%%%%%%%%%%%%%%%%%%%%%%%%%%%%%%%%%%%%%%%%%%%%%%%%%%%%%%%%%%%%%%%%%%%%%%%%%%%%%%
\subsection{Integrated Forward-JWKB Scattering Kernel}
%%%%%%%%%%%%%%%%%%%%%%%%%%%%%%%%%%%%%%%%%%%%%%%%%%%%%%%%%%%%%%%%%%%%%%%%%%%%%%%
From $\mathcal{F}_{T}$ we obtain the un-integrated quantum scattering kernel $\mathcal{S}_{T}$ via
\begin{equation}
	\mathcal{S}_{T}(3, 4|1, 2) = \int \int \int \int \mathrm{d}x_{1} \mathrm{d}x_{2} \mathrm{d}x_{3} \mathrm{d}x_{4} \, \overline{\mathcal{W}}_{O}(3, 4) \mathcal{W}_{I}(1, 2) \mathcal{F}_{T}(3, 4|1, 2).
	\label{STTF}
\end{equation}
The factors $\mathcal{W}_{I}$ and $\overline{\mathcal{W}}_{O}$ account for the asymptotic free massive external states:
\begin{align}
	\mathcal{W}_{I}(1, 2) &= \exp{\left[ \frac{i T_{1}}{4} \left( p_{1}^{2} + m_{1}^{2} \right) + \frac{i T_{2}}{4} \left( p_{2}^{2} + m_{2}^{2} \right) + i x_{1} \cdot p_{1} + i x_{2} \cdot p_{2} \right]}; \\
	\overline{\mathcal{W}}_{O}(3, 4) &= \exp{\left[ \frac{i T_{1}}{4} \left( p_{3}^{2} + m_{3}^{2} \right) + \frac{i T_{2}}{4} \left( p_{4}^{2} + m_{4}^{2} \right) - i x_{3} \cdot p_{3} - i x_{4} \cdot p_{4} \right]}.
\end{align}
Note that a priori we have $m_{1} \neq m_{3} \neq M_{1}$ and $m_{2} \neq m_{4} \neq M_{2}$. That is, the external states are off-shell and the external masses $m_{i}$ are not related to the worldline masses $M_{i}$.

In the forward-JWKB approximation, we use $\mathcal{G}_{T}$ instead of $\mathcal{F}_{T}$ in (\ref{STTF}). In order to perform the integration in (\ref{STTF}), we first make a change of position variables and also introduce the corresponding conjugate momenta,
\begin{align}
	X \equiv \frac{x_{1} + x_{2} + x_{3} + x_{4}}{4}, &\qquad P \equiv p_{3} + p_{4} - p_{1} - p_{2}; \\
	X_{12} \equiv \frac{x_{1} - x_{2} + x_{3} - x_{4}}{2}, &\qquad P_{12} \equiv \frac{p_{3} - p_{1} + p_{2} - p_{4}}{2}; \\
	x_{31} \equiv x_{3} - x_{1}, &\qquad p_{31} \equiv \frac{p_{1} + p_{3}}{2};
	\label{p31} \\
	x_{42} \equiv x_{4} - x_{2}, &\qquad p_{42} \equiv \frac{p_{2} + p_{4}}{2};
	\label{p42}
\end{align}
such that
\begin{equation}
	x_{1} \cdot p_{1} + x_{2} \cdot p_{2} - x_{3} \cdot p_{3} - x_{4} \cdot p_{4} = - X \cdot P - X_{12} \cdot P_{12} - x_{31} \cdot p_{31} - x_{42} \cdot p_{42}.
\end{equation}
The Jacobian from this change of variables is a constant, which we ignore
\begin{equation}
	\mathrm{d}x_{1} \mathrm{d}x_{2} \mathrm{d}x_{3} \mathrm{d}x_{4} \sim \mathrm{d}X \mathrm{d}X_{12} \mathrm{d}x_{31} \mathrm{d}x_{42}.
\end{equation}
In terms of these variables we have
\begin{equation}
\begin{split}
	\overline{\mathcal{W}}_{O}(3, 4) \mathcal{W}_{I}(1, 2) = {}& \exp{\left[ -i X \cdot P -i X_{12} \cdot P_{12} -i x_{31} \cdot p_{31} -i x_{42} \cdot p_{42} \right]} \\
	&\times \exp{\left[ \frac{i T_{1}}{2} p_{31}^{2} + \frac{i T_{1}}{32} \left( 2 P_{1 2} + P \right)^{2} + \frac{i T_{1}}{4} (m_{1}^{2} + m_{3}^{2}) \right]} \\
	&\times \exp{\left[ \frac{i T_{2}}{2} p_{42}^{2} + \frac{i T_{2}}{32} \left( 2 P_{1 2} - P \right)^{2} + \frac{i T_{2}}{4} (m_{2}^{2} + m_{4}^{2}) \right]}.
\end{split}
\end{equation}
Since $\Sigma$ and $V$ have no dependence on $X$, the un-integrated forward-JWKB kernel $\mathcal{G}_{T}$ does not depend on $X$. Thus, the integral over $X$ yields a Dirac delta:
\begin{equation}
	\int \mathrm{d}X \exp{(-i X \cdot P)} = \delta(P).
\end{equation}
This Dirac delta imposes the constraint $P = 0$, which leads to
\begin{equation}
	p_{1} + p_{2} = p_{3} + p_{4}.
\end{equation}
That is, the total external momentum is conserved, as expected from translation invariance. After enforcing $P = 0$, we find
\begin{equation}
	P_{1 2} = p_{3} - p_{1} = p_{2} - p_{4} \quad \Longrightarrow \quad P_{1 2}^{2} = -t.
\end{equation}
Next, we tackle the integration over the $x_{ij}$. The exact integration is nontrivial because of the way that $\rho_{N}$ in $\Sigma_{\text{two}}$ depends on these variables. We make another change of variables:
\begin{equation}
	x_{31} = T_{1} k_{31}, \qquad x_{42} = T_{2} k_{42} \quad \Longrightarrow \quad \mathrm{d}x_{31} \mathrm{d}x_{42} = (T_{1} T_{2})^{D} \mathrm{d}k_{31} \mathrm{d}k_{42}.
\end{equation}
In terms of the $k_{ij}$ we have
\begin{equation}
	\Sigma_{\text{free}} = -\frac{T_{1}}{2} \left(k_{31}^{2} - M_{1}^{2} \right) - \frac{T_{2}}{2} \left( k_{42}^{2} - M_{2}^{2} \right);
\end{equation}
and thus, after integrating over $X$ and enforcing $P = 0$, we find
\begin{equation}
\begin{split}
	\overline{\mathcal{W}}_{O} \mathcal{W}_{I} \exp{(-i \Sigma_{\text{free}})} = {}& \exp{\left[ \frac{i T_{1}}{2} (k_{31} - p_{31})^{2} + \frac{i T_{2}}{2} (k_{42} - p_{42})^{2} - i X_{12} \cdot P_{12} \right]} \\
	&\times \exp{\left[ -\frac{i T_{1}}{4} \left( \frac{t}{2} -m_{1}^{2} - m_{3}^{2} + 2 M_{1}^{2} \right) \right]} \\
	&\times \exp{\left[ -\frac{i T_{2}}{4} \left( \frac{t}{2} -m_{2}^{2} - m_{4}^{2} + 2 M_{2}^{2} \right) \right]}.
\end{split}
\end{equation}
This expression is Gaussian in the $k_{ij}$. The full integrand has the form ``Gaussian $\times$ function''. We resort to stationary methods to approximate the integral over the $k_{ij}$. The stationary point is
\begin{equation}
	\bar{k}_{31} = p_{31}, \qquad \bar{k}_{42} = p_{42}.
\end{equation}
At this stationary point, $\rho_{N}$ becomes a function of the (off-shell) external momenta $p_{ij}$,
\begin{equation}
	\rho_{N} = \frac{\mathcal{K}_{N}}{\sqrt{p_{31}^{2} p_{42}^{2} - (p_{31} \cdot p_{42})^{2}}}, \qquad \mathcal{K}_{N} = \left(p_{3 1} \cdots p_{3 1} \right) \cdot \nu_{N} \cdot \left(p_{4 2} \cdots p_{4 2}\right).
\end{equation}

So far, the un-integrated forward-JWKB scattering kernel looks like
\begin{equation}
\begin{split}
	\mathcal{S}_{T}(3,4|1,2) \approx \delta(P) \int \mathrm{d}X_{12} \, {}& \exp{\left[-i X_{12} \cdot P_{12} + \beta_{N} \rho_{N} \Gamma(\Delta - 1) \left( \frac{2}{B_{1 2}^{2}} \right)^{(\Delta - 1)} \right]} \\
	&\times \exp{\left[ -\frac{i T_{1}}{4} \left( \frac{t}{2} -m_{1}^{2} - m_{3}^{2} + 2 M_{1}^{2} \right) \right]} \\
	&\times \exp{\left[ -\frac{i T_{2}}{4} \left( \frac{t}{2} -m_{2}^{2} - m_{4}^{2} + 2 M_{2}^{2} \right) \right]}.
\end{split}
\label{STTX12}
\end{equation}
We defined $B_{12}$ as the part of $X_{12}$ that is orthogonal to any linear combination of the $x_{ij}$. But the effect of integration over the $x_{ij}$ was to replace $(x_{31}, x_{42})$ with $(T_{1} p_{31}, T_{2} p_{42})$. So now we have the decomposition
\begin{equation}
	X_{12} = B_{12} + T_{1} b_{31} p_{31} + T_{2} b_{42} p_{42}, \qquad B_{12} \cdot p_{31} = 0, \qquad B_{12} \cdot p_{42} = 0.
\end{equation}
The $X_{12}$ volume element becomes
\begin{equation}
	\mathrm{d}X_{12} = T_{1} T_{2} \sqrt{p_{31}^{2} p_{42}^{2} - (p_{31} \cdot p_{42})^{2}} \mathrm{d}B_{12} \mathrm{d}b_{31} \mathrm{d}b_{42};
\end{equation}
which we can write in terms of $\rho_{N}$,
\begin{equation}
	\mathrm{d}X_{12} = T_{1} T_{2} \left( \frac{\mathcal{K}_{N}}{\rho_{N}} \right) \mathrm{d}B_{12} \mathrm{d}b_{31} \mathrm{d}b_{42}.
\end{equation}
Note that
\begin{equation}
	X_{12} \cdot P_{12} = B_{12} \cdot P_{12} + T_{1} b_{31} (p_{31} \cdot P_{12}) + T_{2} b_{42} (p_{42} \cdot P_{12}).
\end{equation}
Since $\Sigma_{\text{two}}$ has no dependence on the $b_{ij}$, integration yields two one-dimensional Dirac deltas:
\begin{align}
	\int \mathrm{d}b_{31} \, \exp{\left[- i T_{1} b_{31} (p_{31} \cdot P_{12}) \right]} &= \frac{1}{T_{1}} \delta(p_{31} \cdot P_{12}); \\
	\int \mathrm{d}b_{42} \, \exp{\left[- i T_{2} b_{42} (p_{42} \cdot P_{12}) \right]} &= \frac{1}{T_{2}} \delta(p_{42} \cdot P_{12}).
\end{align}
We will examine later the constraints that these two Dirac deltas impose. Now the only part of the amplitude that depends on $(T_{1}, T_{2})$ are the second and third lines of (\ref{STTX12}). We must integrate over the moduli in order to obtain the integrated scattering kernel:
\begin{equation}
	\widehat{\mathcal{A}}(3, 4| 1, 2) \equiv \int\limits_{0}^{\infty} \mathrm{d}T_{1} \int\limits_{0}^{\infty} \mathrm{d}T_{2} \, \mathcal{S}_{T}(3, 4| 1, 2).
\end{equation}
Performing the integration over $(T_{1}, T_{2})$ yields
\begin{align}
	\int\limits_{0}^{\infty} \mathrm{d}T_{1} \, \exp{\left[ -\frac{i T_{1}}{4} \left( \frac{t}{2} - m_{1}^{2} - m_{3}^{2} + 2 M_{1}^{2}\right) \right]} &= \frac{8i}{t - 2m_{1}^{2} - 2m_{3}^{2} + 4 M_{1}^{2}}; \\
	\int\limits_{0}^{\infty} \mathrm{d}T_{2} \, \exp{\left[ -\frac{i T_{2}}{4} \left( \frac{t}{2} - m_{2}^{2} - m_{4}^{2} + 2 M_{2}^{2}\right) \right]} &= \frac{8i}{t - 2m_{2}^{2} - 2m_{4}^{2} + 4 M_{2}^{2}}.
\end{align}
The integral over $B_{12}$ remains:
\begin{equation}
	\widehat{\mathcal{A}} = \mathcal{N} \delta(P) \left( \frac{\mathcal{K}_{N}}{\rho_{N}} \right) \int \mathrm{d}B_{12} \, \exp{\left[-i B_{12} \cdot P_{12} + \beta_{N} \rho_{N} \Gamma(\Delta - 1) \left( \frac{2}{B_{1 2}^{2}} \right)^{(\Delta - 1)} \right]}; \label{AHatN}
\end{equation}
where we have collected some terms into an overall factor:
\begin{equation}
	\mathcal{N} \equiv \frac{(8i)^{2} \delta(p_{31} \cdot P_{12}) \delta(p_{42} \cdot P_{12})}{(t - 2m_{1}^{2} - 2m_{3}^{2} + 4 M_{1}^{2})(t - 2m_{2}^{2} - 2m_{4}^{2} + 4 M_{2}^{2})}.
\end{equation}
Before we put the external momenta on-shell, we need to truncate from $\widehat{\mathcal{A}}$ the part that is divergent on-shell.
%%%%%%%%%%%%%%%%%%%%%%%%%%%%%%%%%%%%%%%%%%%%%%%%%%%%%%%%%%%%%%%%%%%%%%%%%%%%%%%
\subsubsection{Truncation of External On-shell States}
%%%%%%%%%%%%%%%%%%%%%%%%%%%%%%%%%%%%%%%%%%%%%%%%%%%%%%%%%%%%%%%%%%%%%%%%%%%%%%%
In quantum field theory, truncation typically involves multiplying the scattering amplitude by a product of inverse propagators $(p_{j}^{2} + m_{j}^{2})$, and taking the limit $p_{j}^{2} \rightarrow - m_{j}^{2}$. This is done in order to remove the part that is divergent on-shell from the scattering amplitude. Since we have four external states, we need to remove four factors from $\widehat{\mathcal{A}}$.

From the relations
\begin{equation}
	p_{31} \cdot P_{12} = \frac{p_{3}^{2} - p_{1}^{2}}{2}, \qquad p_{42} \cdot P_{12} = \frac{p_{2}^{2} - p_{4}^{2}}{2};
\end{equation}
we see that the two Dirac deltas in $\mathcal{N}$ enforce the elasticity constraints
\begin{equation}
	p_{1}^{2} = p_{3}^{2}, \qquad p_{2}^{2} = p_{4}^{2}.
\end{equation}
Furthermore, on-shell we have
\begin{equation}
	p_{31}^{2} = \frac{t - 2 m_{1}^{2} - 2m_{3}^{2}}{4}, \qquad p_{42}^{2} = \frac{t - 2 m_{2}^{2} - 2m_{4}^{2}}{4};
\end{equation}
so then the denominators in $\mathcal{N}$ become
\begin{equation}
	\frac{8i}{t - 2m_{1}^{2} - 2m_{3}^{2} + 4 M_{1}^{2}} = \frac{2i}{p_{31}^{2} + M_{1}^{2}}, \qquad \frac{8i}{t - 2m_{2}^{2} - 2m_{4}^{2} + 4 M_{2}^{2}} = \frac{2i}{p_{42}^{2} + M_{2}^{2}}.
\end{equation}
These two factors have the form of (free) Feynman propagators for particles with momenta $(p_{31}, p_{42})$ and masses $(M_{1}, M_{2})$; they diverge when $p_{31}^{2} \rightarrow -M_{1}^{2}$ and $p_{42}^{2} \rightarrow -M_{2}^{2}$. We can think of these limits as ways to relate the external on-shell momenta to the ``internal'' masses $(M_{1}, M_{2})$:
\begin{equation}
	M_{1}^{2} = \frac{2m_{1}^{2} + 2m_{3}^{2} - t}{4}, \qquad M_{2}^{2} = \frac{2m_{2}^{2} + 2m_{4}^{2} - t}{4}.
\end{equation}
Note that these relations satisfy
\begin{equation}
	2M_{1}^{2} + 2M_{2}^{2} = m_{1}^{2} + m_{2}^{2} + m_{3}^{2} + m_{4}^{2} - t = s + u.
\end{equation}
After we enforce the elasticity constraints, we find
\begin{equation}
	M_{1}^{2} = m_{1}^{2} \left( 1 - \frac{t}{4 m_{1}^{2}} \right), \qquad M_{2}^{2} = m_{2}^{2} \left( 1 - \frac{t}{4 m_{2}^{2}} \right);
\end{equation}
and thus, in the forward-JWKB approximation (\ref{fJWKBLimit}) we have $M_{1} \approx m_{1}$ and $M_{2} \approx m_{2}$. The upshot of this discussion is that we can think of the factors in $\mathcal{N}$ as restricting the external momenta to be on-shell. Truncation is achieved by simply dropping $\mathcal{N}$ from (\ref{AHatN}). Thus, the \textit{truncated, on-shell, forward-JWKB scattering amplitude} $\mathcal{A}$ is given by
\begin{equation}
	\mathcal{A} = \delta(P) \left( \frac{\mathcal{K}_{N}}{\rho_{N}} \right) \int \mathrm{d}B_{12} \, \exp{\left[-i B_{12} \cdot P_{12} + \beta_{N} \rho_{N} \Gamma(\Delta - 1) \left( \frac{2}{B_{1 2}^{2}} \right)^{(\Delta - 1)} \right]}.
	\label{AHND}
\end{equation}
Recall that $B_{12}$ is a vector in $D$ dimensions subjected to two orthogonality constraints. Thus, the $B_{12}$ integral is over a $(D-2)$-dimensional volume. In \S\ref{sec4} and \S\ref{sec5} we evaluate this integral in $D = 3$ and $D = 4$.