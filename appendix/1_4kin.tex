\section{Four-point Kinematics\label{app1}}
%%%%%%%%%%%%%%%%%%%%%%%%%%%%%%%%%%%%%%%%%%%%%%%%%%%%%%%%%%%%%%%%%%%%%%%%%%%%%%%%%%%%%%%%%
We study the scattering of two non-identical massive scalar particles. The external states are labeled such that the $s$-channel process is elastic:
\begin{equation}
	A(p_{1}) + B(p_{2}) \longrightarrow A(p_{3}) + B(p_{4}) \label{sChannel}
\end{equation}
Note that the $u$-channel process is also elastic:
\begin{equation}
	A(p_{1}) + \bar{B}(\bar{p}_{2}) \longrightarrow A(p_{3}) + \bar{B}(\bar{p}_{4}) \label{uChannel}
\end{equation}
but the $t$-channel process is inelastic:
\begin{equation}
	A(p_{1}) + \bar{A}(\bar{p}_{2}) \longrightarrow \bar{B}(\bar{p}_{3}) + B(p_{4}) \label{tChannel}
\end{equation}
We will only consider the amplitude for the (\ref{sChannel}) process. The amplitude for the (\ref{uChannel}) process follows after setting $\bar{p}_{2} = -p_{4}$ and $\bar{p}_{4} = -p_{2}$, and the amplitude for the (\ref{tChannel}) process follows after setting $\bar{p}_{2} = -p_{3}$ and $\bar{p}_{3} = -p_{2}$.

The four external energy-momentum vectors satisfy the on-shell constraints
\begin{equation}
	p_{j}^{2} = - m_{j}^{2};
\end{equation}
the elasticity constraints
\begin{equation}
	p_{1}^{2} = p_{3}^{2}, \qquad p_{2}^{2} = p_{4}^{2} \quad \Longrightarrow \quad m_{1} = m_{3}, \qquad m_{2} = m_{4}; \label{elas}
\end{equation}
and also the conservation constraint
\begin{equation}
	p_{1} + p_{2} = p_{1} + p_{4}
\end{equation}
We use the familiar Mandelstam energy-momentum invariants
\begin{equation}
	s = -(p_{1} + p_{2})^{2}, \qquad t = -(p_{1} - p_{3})^{2}, \qquad u = -(p_{1} - p_{4})^{2}
\end{equation}
which satisfy
\begin{equation}
	s + t + u = 2 m_{1}^{2} + 2 m_{2}^{2}
\end{equation}
Note that $s$ and $u$ carry data from both particles, while $t$ carries data from a single particle.

In the center-of-momentum frame, we write the energy-momentum vectors as
\begin{equation}
	p_{1} = (E_{1}, \mathbf{p}_{1}), \qquad p_{2} = (E_{2}, -\mathbf{p}_{1}), \qquad p_{3} = (E_{3}, \mathbf{p}_{3}), \qquad p_{4} = (E_{4}, -\mathbf{p}_{3})
\end{equation}
It is easy to show that, in terms of $s$ and the masses $(m_{1}, m_{2})$, the magnitude of the spatial vectors $(\mathbf{p}_{1}, \mathbf{p}_{3})$ are given by
\begin{equation}
	|\mathbf{p}_{1}| = |\mathbf{p}_{3}| = \frac{\sqrt{\Lambda_{12}(s)}}{2 \sqrt{s}}, \qquad \Lambda_{12}(s) \equiv \left[s - (m_{1} - m_{2})^{2} \right] \left[s - (m_{1} + m_{2})^{2} \right] \label{2Momenta}
\end{equation}
and the energies of the external states are given by
\begin{equation}
	E_{1} = E_{3} = \frac{s + (m_{1} - m_{2})(m_{1} + m_{2})}{2 \sqrt{s}}, \qquad E_{2} = E_{4} = \frac{s - (m_{1} - m_{2})(m_{1} + m_{2})}{2 \sqrt{s}} \label{4Energies}
\end{equation}
A relativistic particle with mass $m \geq 0$ and energy $E \geq m$ has a speed $|\mathbf{v}| \leq 1$ given by
\begin{equation}
	|\mathbf{v}| = \frac{\sqrt{E^{2} - m^{2}}}{E}
\end{equation}
Thus, in the center-of-momentum frame the external states have speeds
\begin{equation}
\begin{split}
	|\mathbf{v}_{1}| &= |\mathbf{v}_{3}| = \frac{\sqrt{\Lambda_{12}(s)}}{s + (m_{1} - m_{2})(m_{1} + m_{2})} \\
	|\mathbf{v}_{2}| &= |\mathbf{v}_{4}| = \frac{\sqrt{\Lambda_{12}(s)}}{s - (m_{1} - m_{2})(m_{1} + m_{2})}
\end{split} \label{4Speeds}
\end{equation}
and rapidities $\varphi_{j} \equiv \operatorname{artanh}{(|\mathbf{v}_{j}|)}$ given by
\begin{equation}
\begin{split}
	\varphi_{1} &= \varphi_{3} = \frac{1}{2} \log{\left[ \frac{(m_{1} - m_{2})(m_{1} + m_{2}) + s + \sqrt{\Lambda_{12}(s)}}{(m_{1} - m_{2})(m_{1} + m_{2}) + s - \sqrt{\Lambda_{12}(s)}} \right]} \\
	\varphi_{2} &= \varphi_{4} = \frac{1}{2} \log{\left[ \frac{(m_{1} - m_{2})(m_{1} + m_{2}) - s - \sqrt{\Lambda_{12}(s)}}{(m_{1} - m_{2})(m_{1} + m_{2}) - s + \sqrt{\Lambda_{12}(s)}} \right]}
\end{split} \label{4Rapidities}
\end{equation}
Note that the sum of the incoming rapidities gives
\begin{equation}
	\varphi_{1} + \varphi_{2} = \frac{1}{2} \log{\left[ \frac{m_{1}^{2} + m_{2}^{2} - s - \sqrt{\Lambda_{12}(s)}}{m_{1}^{2} + m_{2}^{2} - s + \sqrt{\Lambda_{12}(s)}} \right]}
\end{equation}
Using
\begin{equation}
	\frac{(m_{1} + m_{2})^{2} - s + \sqrt{\Lambda_{12}(s)}}{(m_{1} + m_{2})^{2} - s - \sqrt{\Lambda_{12}(s)}} = \frac{2m_{1}m_{2}}{m_{1}^{2} + m_{2}^{2} - s + \sqrt{\Lambda_{12}(s)}} = \frac{m_{1}^{2} + m_{2}^{2} - s - \sqrt{\Lambda_{12}(s)}}{2m_{1}m_{2}}
\end{equation}
we can also write
\begin{equation}
	\varphi_{1} + \varphi_{2} = \log{\left[ \frac{(m_{1} + m_{2})^{2} - s + \sqrt{\Lambda_{12}(s)}}{(m_{1} + m_{2})^{2} - s - \sqrt{\Lambda_{12}(s)}} \right]} \label{rap12}
\end{equation}
The cosine of the angle between $\mathbf{p}_{1}$ and $\mathbf{p}_{3}$ is given by
\begin{equation}
	z_{s} \equiv \cos(\theta_{s}) = \frac{\mathbf{p}_{1} \cdot \mathbf{p}_{3}}{|\mathbf{p}_{1}| |\mathbf{p}_{3}|} = \frac{(m_{1} - m_{2})^{2} (m_{1} + m_{2})^{2} - s (u - t)}{(m_{1} - m_{2})^{2} (m_{1} + m_{2})^{2} - s (u + t)} \label{z13rational}
\end{equation}
The angle $\theta_{s}$ is known as the scattering angle.
%%%%%%%%%%%%%%%%%%%%%%%%%%%%%%%%%%%%%%%%%%%%%%%%%%%%%%%%%%%%%%%%%%%%%%%%%%%%%%%%%%%%%%%%%
\subsection{Physical Scattering Region}
%%%%%%%%%%%%%%%%%%%%%%%%%%%%%%%%%%%%%%%%%%%%%%%%%%%%%%%%%%%%%%%%%%%%%%%%%%%%%%%%%%%%%%%%%
Requiring each of the energies in (\ref{4Energies}) to be real and non-negative leads to the condition
\begin{equation}
	s \geq |m_{1} - m_{2}| (m_{1} + m_{2})
\end{equation}
Indeed, the same requirements on the magnitudes in (\ref{2Momenta}) leads to a stronger condition:
\begin{equation}
	\Lambda_{12}(s) \geq 0 \quad \Longrightarrow \quad s \geq (m_{1} + m_{2})^{2} > |m_{1} - m_{2}| (m_{1} + m_{2})
\end{equation}
We must also require $z_{s}$ in (\ref{z13rational}) to satisfy
\begin{equation}
	{-1} \leq z_{s} \leq 1
\end{equation}
This is equivalent to the conditions
\begin{equation}
	t \leq 0, \qquad s u \leq (m_{1} - m_{2})^{2} (m_{1} + m_{2})^{2}
\end{equation}
Thus, the physical scattering region is defined by
\begin{equation}
	s \geq (m_{1} + m_{2})^{2}, \qquad t \leq 0, \qquad s u \leq (m_{1} - m_{2})^{2} (m_{1} + m_{2})^{2}
\end{equation}
Since these relations involve Lorentz invariants, they hold on any reference frame related to the center-of-momentum frame by a Lorentz transformation.